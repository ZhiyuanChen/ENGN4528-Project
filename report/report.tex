% last updated in April 2002 by Antje Endemann
% Based on CVPR 07 and LNCS, with modifications by DAF, AZ and elle, 2008 and AA, 2010, and CC, 2011; TT, 2014; AAS, 2016

\documentclass[runningheads]{llncs}
\usepackage{graphicx}
\usepackage{amsmath,amssymb} % define this before the line numbering.
\usepackage{ruler}
\usepackage{color}
\usepackage[width=122mm,left=12mm,paperwidth=146mm,height=193mm,top=12mm,paperheight=217mm]{geometry}
\begin{document}
% \renewcommand\thelinenumber{\color[rgb]{0.2,0.5,0.8}\normalfont\sffamily\scriptsize\arabic{linenumber}\color[rgb]{0,0,0}}
% \renewcommand\makeLineNumber {\hss\thelinenumber\ \hspace{6mm} \rlap{\hskip\textwidth\ \hspace{6.5mm}\thelinenumber}}
% \linenumbers
\pagestyle{headings}
\mainmatter
\def\ECCV18SubNumber{31}

\title{Self-Driving Assistant in Computer Simulation 
Environment}

\titlerunning{ENGN4528 Group \ECCV18SubNumber}

\authorrunning{ENGN4528 Group \ECCV18SubNumber}

\author{Zhiyuan Chen, Xingyuan Xu, Qiusi Xiang, Bisyri 
Hisham, Kavinenh Mohanraj}
\institute{Australian National University}


\maketitle

\begin{abstract}
This project implement a self-driving assistant in computer 
simulation environment with Spatial CNN and Mask RCNN on a 
distributed system.
\dots
\keywords Self-Driving, Spatial CNN, Mask RCNN, Distributed 
System
\end{abstract}


\section{Introduction}
Self-driving technologies had been used in airplanes (known 
as AP for Auto Pilot), and trains (known as ATO for 
Automatic Train Operation) for decades. However, as the 
road traffic is far more complex, self-driving cars have 
never been in commerical use. Thanks to the development of 
Machine Learning, Computer Vision and most importantly the 
hardwares, self-driving cars does not seem to be impossible 
today. Thus, we designed a self-driving assistant program 
in the computer simulation environment.

The best performed models were chosen for this project to 
achieve the best performance. 

\section{Pre Process}
\subsection{Screenshot}
Taking a screenshot can be harder than it looks. 
ImageGrab.grab() from PIL is a good approach. However, 
it costs 300-400ms to take a single screenshot which is 
obviously unacceptable. We test many different functions 
from different package, as a result, Python-MSS is much 
faster than any other package in the comparison with a 
average speed of 60fps.

\subsection{Transmission}
Transmitting images between each cluster is another key 
point. The delayness can be devided to two part, the 
encode(serialize)/decode(deserialize) time and the 
transmission time. We tried three different ways to encode 
our image, pickle, json, and imencode. 

\section{Process}

\subsection{Lane Line Detection}

\subsection{Obstacle Detection}


\section{Post Process}


\section{Distributed Scheduling}

\subsection{Containerization}
Containerization has been widely used in many industries. 
Compare to traditional virtual machine, the performance 
and resources loss are reduced to a large extent benefits 
from removing the guest OS and hardware virtualization. 
Docker, as the most popular container platform, are used here
in this project to containerlize our algorithm so that 


\subsection{Container Orchestration}

\section{Conclusions}


\clearpage

\bibliographystyle{splncs}
\bibliography{egbib}
\end{document}
